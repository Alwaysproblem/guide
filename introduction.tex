\section{Introduction}\label{sec:introduction}

The ``Potsdam Answer Set Solving Collection'' (Potassco;~\cite{potassco,gekakaosscsc11a,gekakasc12a})
gathers a variety of tools for Answer Set Programming (ASP;~%
\cite{ankolisc05a,baral02a,breitr11a,gelfond08a,gelkah14a,gelleo02a,lifschitz02a,martru99a,niemela99a}),
including grounder \gringo, solver \clasp, and their combination within the integrated ASP system \clingo.
Their common goal is to provide means enabling users 
to rapidly solve difficult computational problems in ASP,
a declarative programming paradigm based on
logic programs and their answer sets.

To be more precise, 
we focus in this guide on ASP and thus the computation of the so-called stable models of a logic program~\cite{gellif88b}.
Moreover, \clasp\ can be used as a full-fledged SAT, MaxSAT, or PB solver (see~\cite{SATHandbook}), 
accepting propositional CNF formulas in (extended) DIMACS format as well as PB formulas in OPB and WBO format.

\subsection{Download and Installation}

The Potassco tools \gringo, \clasp, and \clingo\
are written in \texttt{C++} and published under GNU General Public License(s)~\cite{GNUgpl}.
Source packages as well as precompiled binaries for Linux, MacOS, and Windows
are available at~\cite{potassco}.
For building some of the tools from sources,
please download the most recent source package, consult the
included \code{README} % or \code{INSTALL} text file, respectively.
file,
and make sure that the machine to build on has all
required software installed.
If you still encounter problems in the building process,
please 
consult the support pages at~\cite{potassco}
or use the Potassco mailing list:
\href{mailto:potassco-users@lists.sourceforge.net}{\texttt{potassco-users@lists.sourceforge.net}}.

After downloading (and possibly building) a tool,
one can check whether everything works fine by invoking the tool
with flag \code{--version} (to get version information) or
with flag \code{--help} (to see the available command line options).
For instance, assuming that a binary called \gringo\ is in the path
(similarly with the other tools),
the following command line calls should be responded by \gringo:
%
\begin{lstlisting}[numbers=none]
gringo --version
gringo --help
\end{lstlisting}

Note that \gringo, \clasp, and \clingo\ 
run on the command line (Linux shell, Windows command prompt, or the like).
To facilitate invoking them, their binaries can be ``installed''
simply by putting them into some directory in the system path.
In an invocation,
one usually provides the file names of input (text) files 
as arguments to either \gringo, \clingo, or \iclingo,
while the output of \gringo\ is typically piped into \clasp.
Thus, the standard invocation schemes are as follows:
\begin{lstlisting}[numbers=none]
gringo  [ options | files ] | clasp [ options | number ]
clingo  [ options | files | number ]
\end{lstlisting}
A numerical argument provided to either \clasp\ or \clingo\
determines the maximum number of answer sets to be computed,
where \code{0} means ``compute all answer sets.''
By default, only one answer set is computed (if it exists).

\subsection{Outline}

This guide introduces the fundamentals of using
\gringo, \clasp, and \clingo.
In particular, it aims at enabling the reader to benefit from them
by significantly reducing the ``time to solution'' on difficult computational problems.
% The outline is as follows.
To this end,
Section~\ref{sec:quickstart}
provides an introductory example 
that serves both as a prototype of problem modeling using logic programs
and also as an appetizer of the modeling language of \gringo.
The main part of this document, Section~\ref{sec:language},
is dedicated to the input languages of our tools,
where Section~\ref{subsec:lang:gringo}
details the joint input language of \gringo\ and \clingo.
In Section~\ref{subsec:lang:clingo}, it is extended with
directives enabling the integration of \python\ and \lua\ capacities with \clingo.
The input language of \clasp\ is discussed only briefly in Section~\ref{subsec:lang:clasp},
as it matches the numerical output format of \gringo\ and
is not supposed to be written directly by a user.
For %illustrating the application of our tools,
further illustration,
Section~\ref{sec:examples} describes how three well-known example problems
can be solved with our tools.
Practical aspects are also in the focus of Section~\ref{sec:options} and~\ref{sec:error:warn},
where we elaborate and give some hints on the available command line options
as well as input-related errors and warnings. % that may be reported.
We conclude with a summary in Section~\ref{sec:future}.

%For a more theoretical background, the interested reader is referred
%to Appendix~\ref{sec:background} where technical details are covered.

For readers familiar with \lparse~\cite{lparseManual}%
\footnote{A grounder that constitutes the traditional front end of solver \smodels~\cite{siniso02a})}
or the \gringo~3 series,,
Appendices~\ref{sec:lparse} and~\ref{sec:gringo:tri}
% \comment{Appendix not updated yet}
list the most prominent differences to our current tools.
Otherwise, \gringo\ and \clingo\ series~4 should accept most inputs recognized by \lparse\ and \gringo~3,
while the input of solver \clasp\ can also be generated by \lparse\ instead of \gringo.
Throughout this document, we provide illustrative examples.
Many of them can actually be run, and instructions on how to accomplish this
(or sometimes meta-remarks)
are provided in margin boxes, like the one on the right,%
\marginlabel{I am a margin box.
             Me and my friends provide you
             with hints.
             When I write ``\code{\char`\\},''
             it means that I break a continuous line
             to stay within margins.}
where an occurrence of ``\code{\char`\\}''
usually means that text in a command line, broken for space reasons, is actually continuous.

We omit any formal aspects of ASP and refer the interested reader for this to the literature~\cite{baral02a,gelkah14a,gekakasc12a}.

After all these preliminaries, it is time to start our guided tour
through the main Potassco~\cite{potassco} tools.
We hope that you will find it enjoyable and helpful!

%%% Local Variables: 
%%% mode: latex
%%% TeX-master: "guide"
%%% End: 
