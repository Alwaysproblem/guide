\section{Multi-shot Solving}\label{sec:multi}

This section is not yet ready for publishing
and will be included in one of the forthcoming editions of this guide.

Information on multi-shot solving with \clingo\ can be obtained at the following references.

\begin{itemize}
\item \cite{kascwa17a,gekakasc17a}
\item \code{/examples/clingo/} in \gringo/\clingo\ distribution
\item API reference
  \begin{itemize}
  \item Version 4.5.4:  \url{http://potassco.sourceforge.net/gringo.html}
  \item Series 5:       \url{http://potassco.org/clingo}
  \end{itemize}
\end{itemize}

% \subsection{Incremental Solving}
% \label{sec:isolving}

% \begin{itemize}
% \item \lstinline{#step}
% \item \lstinline{#check}
%   \begin{itemize}
%   \item starts at 0
% (For example, this is usful in planning to capture situationss where the goal is already satisfied in te initial situation)
%   \item has external \lstinline{query/1}
%   \end{itemize}
% \end{itemize}

% \lstinputlisting[caption={\lstinline{gringo/examples/clingo/iclingo/incmode-py.lp}}]{examples/incmode-py.lp}

% \lstinputlisting[caption={\lstinline{gringo/examples/clingo/iclingo/incmode-lua.lp}}]{examples/incmode-lua.lp}

% \lstinline{#include <incmode>.}

%%% Local Variables: 
%%% mode: latex
%%% TeX-master: "guide"
%%% End: 
